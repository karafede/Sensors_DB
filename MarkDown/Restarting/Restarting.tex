%  LaTeX support: latex@mdpi.com
%  In case you need support, please attach all files that are necessary for compiling as well as the log file, and specify the details of your LaTeX setup (which operating system and LaTeX version / tools you are using).

%=================================================================
\documentclass[sensors,article,submit,moreauthors,pdftex]{mdpi}

% If you would like to post an early version of this manuscript as a preprint, you may use preprint as the journal and change 'submit' to 'accept'. The document class line would be, e.g., \documentclass[preprints,article,accept,moreauthors,pdftex]{mdpi}. This is especially recommended for submission to arXiv, where line numbers should be removed before posting. For preprints.org, the editorial staff will make this change immediately prior to posting.

%% Some pieces required from the pandoc template
\providecommand{\tightlist}{%
  \setlength{\itemsep}{0pt}\setlength{\parskip}{4pt}}
\setlist[itemize]{leftmargin=*,labelsep=5.8mm}
\setlist[enumerate]{leftmargin=*,labelsep=4.9mm}

\usepackage{longtable}

% see https://stackoverflow.com/a/47122900

%--------------------
% Class Options:
%--------------------
%----------
% journal
%----------
% Choose between the following MDPI journals:
% acoustics, actuators, addictions, admsci, aerospace, agriculture, agriengineering, agronomy, algorithms, animals, antibiotics, antibodies, antioxidants, applsci, arts, asc, asi, atmosphere, atoms, axioms, batteries, bdcc, behavsci , beverages, bioengineering, biology, biomedicines, biomimetics, biomolecules, biosensors, brainsci , buildings, cancers, carbon , catalysts, cells, ceramics, challenges, chemengineering, chemistry, chemosensors, children, cleantechnol, climate, clockssleep, cmd, coatings, colloids, computation, computers, condensedmatter, cosmetics, cryptography, crystals, dairy, data, dentistry, designs , diagnostics, diseases, diversity, drones, econometrics, economies, education, electrochem, electronics, energies, entropy, environments, epigenomes, est, fermentation, fibers, fire, fishes, fluids, foods, forecasting, forests, fractalfract, futureinternet, futurephys, galaxies, games, gastrointestdisord, gels, genealogy, genes, geohazards, geosciences, geriatrics, hazardousmatters, healthcare, heritage, highthroughput, horticulturae, humanities, hydrology, ijerph, ijfs, ijgi, ijms, ijns, ijtpp, informatics, information, infrastructures, inorganics, insects, instruments, inventions, iot, j, jcdd, jcm, jcp, jcs, jdb, jfb, jfmk, jimaging, jintelligence, jlpea, jmmp, jmse, jnt, jof, joitmc, jpm, jrfm, jsan, land, languages, laws, life, literature, logistics, lubricants, machines, magnetochemistry, make, marinedrugs, materials, mathematics, mca, medicina, medicines, medsci, membranes, metabolites, metals, microarrays, micromachines, microorganisms, minerals, modelling, molbank, molecules, mps, mti, nanomaterials, ncrna, neuroglia, nitrogen, notspecified, nutrients, ohbm, particles, pathogens, pharmaceuticals, pharmaceutics, pharmacy, philosophies, photonics, physics, plants, plasma, polymers, polysaccharides, preprints , proceedings, processes, proteomes, psych, publications, quantumrep, quaternary, qubs, reactions, recycling, religions, remotesensing, reports, resources, risks, robotics, safety, sci, scipharm, sensors, separations, sexes, signals, sinusitis, smartcities, sna, societies, socsci, soilsystems, sports, standards, stats, surfaces, surgeries, sustainability, symmetry, systems, technologies, test, toxics, toxins, tropicalmed, universe, urbansci, vaccines, vehicles, vetsci, vibration, viruses, vision, water, wem, wevj

%---------
% article
%---------
% The default type of manuscript is "article", but can be replaced by:
% abstract, addendum, article, benchmark, book, bookreview, briefreport, casereport, changes, comment, commentary, communication, conceptpaper, conferenceproceedings, correction, conferencereport, expressionofconcern, extendedabstract, meetingreport, creative, datadescriptor, discussion, editorial, essay, erratum, hypothesis, interestingimages, letter, meetingreport, newbookreceived, obituary, opinion, projectreport, reply, retraction, review, perspective, protocol, shortnote, supfile, technicalnote, viewpoint
% supfile = supplementary materials

%----------
% submit
%----------
% The class option "submit" will be changed to "accept" by the Editorial Office when the paper is accepted. This will only make changes to the frontpage (e.g., the logo of the journal will get visible), the headings, and the copyright information. Also, line numbering will be removed. Journal info and pagination for accepted papers will also be assigned by the Editorial Office.

%------------------
% moreauthors
%------------------
% If there is only one author the class option oneauthor should be used. Otherwise use the class option moreauthors.

%---------
% pdftex
%---------
% The option pdftex is for use with pdfLaTeX. If eps figures are used, remove the option pdftex and use LaTeX and dvi2pdf.

%=================================================================
\firstpage{1}
\makeatletter
\setcounter{page}{\@firstpage}
\makeatother
\pubvolume{xx}
\issuenum{1}
\articlenumber{5}
\pubyear{2019}
\copyrightyear{2019}
%\externaleditor{Academic Editor: name}
\history{Received: date; Accepted: date; Published: date}
\updates{yes} % If there is an update available, un-comment this line

%% MDPI internal command: uncomment if new journal that already uses continuous page numbers
%\continuouspages{yes}

%------------------------------------------------------------------
% The following line should be uncommented if the LaTeX file is uploaded to arXiv.org
%\pdfoutput=1

%=================================================================
% Add packages and commands here. The following packages are loaded in our class file: fontenc, calc, indentfirst, fancyhdr, graphicx, lastpage, ifthen, lineno, float, amsmath, setspace, enumitem, mathpazo, booktabs, titlesec, etoolbox, amsthm, hyphenat, natbib, hyperref, footmisc, geometry, caption, url, mdframed, tabto, soul, multirow, microtype, tikz

%=================================================================
%% Please use the following mathematics environments: Theorem, Lemma, Corollary, Proposition, Characterization, Property, Problem, Example, ExamplesandDefinitions, Hypothesis, Remark, Definition
%% For proofs, please use the proof environment (the amsthm package is loaded by the MDPI class).

%=================================================================
% Full title of the paper (Capitalized)
\Title{Calibration of electrochemical sensors for air pollution monitoring
using the AirSensEUR box -- Laboratory experiments}

% Authors, for the paper (add full first names)
\Author{F. Karagulian$^{1,\ddagger}$\href{https://orcid.org/0000-0003-0518-0955}{\orcidicon}, L. Spinelle2$^{2,\ddagger}$\href{https://orcid.org/0000-0002-5832-3588}{\orcidicon}, F. A. Kotsev$^{3,\ddagger}$\href{https://orcid.org/0000-0003-0411-741X}{\orcidicon}, M. Signorini$^{4,\ddagger}$\href{https://orcid.org/0000-0003-0518-0955}{\orcidicon}, M. Gerboles$^{5,\ddagger,*}$\href{https://orcid.org/0000-0002-7015-1627}{\orcidicon}}

% Authors, for metadata in PDF
\AuthorNames{F. Karagulian, L. Spinelle2, F. A. Kotsev, M. Signorini, M. Gerboles}

% Affiliations / Addresses (Add [1] after \address if there is only one affiliation.)
\address{%
$^{1}$ \quad European Commision, Joint Research Centre, Ispra, Italy; \href{mailto:federico.karagulian@ec.europa.eu}{\nolinkurl{federico.karagulian@ec.europa.eu}}\\
$^{2}$ \quad Institut national de l'Environnement Industriel et des Risques (INERIS)
Parc technologique Alata, BP 2, F-60550 Verneuil-en-Halatte, France; \href{mailto:laurent.spinelle@ineris.fr}{\nolinkurl{laurent.spinelle@ineris.fr}}\\
$^{3}$ \quad European Commision, Joint Research Centre, Ispra, Italy; \href{mailto:alexander.kotsev@ec.europa.eu}{\nolinkurl{alexander.kotsev@ec.europa.eu}}\\
$^{4}$ \quad Liberaintentio Srl, Malnate 21046, Italy; \href{mailto:marco.signorini@liberaintentio.com}{\nolinkurl{marco.signorini@liberaintentio.com}}\\
$^{5}$ \quad European Commision, Joint Research Centre, Ispra, Italy; \href{mailto:michel.gerboles@ec.europa.eu}{\nolinkurl{michel.gerboles@ec.europa.eu}}\\
}
% Contact information of the corresponding author
\corres{Correspondence: \href{mailto:michel.gerboles@ec.europa.eu}{\nolinkurl{michel.gerboles@ec.europa.eu}};
Tel.: +39-0332-78-5652}

% Current address and/or shared authorship
\firstnote{Current address: Updated affiliation}
\secondnote{These authors contributed equally to this work.}






% The commands \thirdnote{} till \eighthnote{} are available for further notes

% Simple summary
\simplesumm{A Simple summary goes here.}

% Abstract (Do not insert blank lines, i.e. \\)
\abstract{A single paragraph of about 200 words maximum. For research articles,
abstracts should give a pertinent overview of the work. We strongly
encourage authors to use the following style of structured abstracts,
but without headings: 1) Background: Place the question addressed in a
broad context and highlight the purpose of the study; 2) Methods:
Describe briefly the main methods or treatments applied; 3) Results:
Summarize the article's main findings; and 4) Conclusion: Indicate the
main conclusions or interpretations. The abstract should be an objective
representation of the article, it must not contain results which are not
presented and substantiated in the main text and should not exaggerate
the main conclusions.}

% Keywords
\keyword{temperature effect; humidity effect; interference; drift.}

% The fields PACS, MSC, and JEL may be left empty or commented out if not applicable
%\PACS{J0101}
%\MSC{}
%\JEL{}

%%%%%%%%%%%%%%%%%%%%%%%%%%%%%%%%%%%%%%%%%%
% Only for the journal Diversity
%\LSID{\url{http://}}

%%%%%%%%%%%%%%%%%%%%%%%%%%%%%%%%%%%%%%%%%%
% Only for the journal Applied Sciences:
%\featuredapplication{Authors are encouraged to provide a concise description of the specific application or a potential application of the work. This section is not mandatory.}
%%%%%%%%%%%%%%%%%%%%%%%%%%%%%%%%%%%%%%%%%%

%%%%%%%%%%%%%%%%%%%%%%%%%%%%%%%%%%%%%%%%%%
% Only for the journal Data:
%\dataset{DOI number or link to the deposited data set in cases where the data set is published or set to be published separately. If the data set is submitted and will be published as a supplement to this paper in the journal Data, this field will be filled by the editors of the journal. In this case, please make sure to submit the data set as a supplement when entering your manuscript into our manuscript editorial system.}

%\datasetlicense{license under which the data set is made available (CC0, CC-BY, CC-BY-SA, CC-BY-NC, etc.)}

%%%%%%%%%%%%%%%%%%%%%%%%%%%%%%%%%%%%%%%%%%
% Only for the journal Toxins
%\keycontribution{The breakthroughs or highlights of the manuscript. Authors can write one or two sentences to describe the most important part of the paper.}

%\setcounter{secnumdepth}{4}
%%%%%%%%%%%%%%%%%%%%%%%%%%%%%%%%%%%%%%%%%%


\begin{document}
%%%%%%%%%%%%%%%%%%%%%%%%%%%%%%%%%%%%%%%%%%

\section{Version}\label{version}

This Rmd-skeleton uses the mdpi Latex template published 2019/02.
However, the official template gets more frequently updated than the
`rticles' package. Therefore, please make sure prior to paper
submission, that you're using the most recent .cls, .tex and .bst files
(available \href{http://www.mdpi.com/authors/latex}{here}).

\section{Introduction}\label{introduction}

In the last years an incrusing number of studies describing several
methods for the calibration of electrochemical low-cost sensors have
been pusblished. The majority of these calibration methods by field
experiments with co-location of sensor systems and reference analysers
sited at Air Quality Monitoring Stations (AQMS) (references needed).
Converseley, little or no laboratory experiments under controlled
conditions are generally carried out. The aim of field test experiments
is to establish the relationship between sensor data and reference
measurements possibly taking into consideration co-variates that affects
sensor responses (Reference needed) and later prediction of sensor data
at other sites than the one use for calibration. Among this type of
model equations used for calibration, one can distinghuish between
emprical (multi-)linear approaches \citep{spinelle_field_2015} and
machine learning methods
\citep{de_vito_calibrating_2018, spinelle_field_2015}. It has become an
established technical methodology to improve the data quality of such
sensor systems by co-locating them at traditional air quality monitoring
stations equipped with reference instrumentation and field-calibrating
individual units using various statistical techniques. Methods range
from (multi-)linear regression to more complex statistical techniques,
often using additional predictor variables such as air temperature or
relative humidity (e.g.~Spinelle et al., 2017), and occasionally data
not actually measured by the sensor system itself (e.g.~station
observations or model output). that is data fishing, the tests in the
data sheets of manufacturer can help to correctly fish.

Several laboratories studies have shown that low cost sensor are
strongly affected from changes of several Variables either as
cross-sensitivies (reference) or meteorological parameters, mainly
temperature and humidity However, in field the variables affecting
sensor measurements are not independent. Major co-linearities are
common, for example between CO, NO and PM or anti-correlation between
temperature and humidity. The majority of statistical tools, e. g.
multilinear regression or \ldots{}, are affected by collinearities of xs
making the decision of taking into account affecting parameters very
much erratic. Including type 1 and type 2, it is becoming impossible to
say which parameter drives the system

The trend is towards artificial intelligence ANN and random forest seems
not to be able to manage change of sampling sites De vito and Hueglin).

Is it possible to use simple deterministic model Are these model similar
at the same place for several ASE. -\textgreater{} pre calibration set

What if we change of place? Type of site Sensitivity analysis effect of
parameter variability of measurement uncertainty Best period for
calibration, rolling calibration period and \(R^2\) of calibration
Principle of electrochemical sensor Position of the auxiliary electrode:
it remains within the electrolyte while the working electrode is exposed
to air. The electrolyte is a gel Effect Are the effect temperature,
relative humidity, cross sensitivities additive? Using the results of
laboratory experiments, (Mueller et al., 2017) concluded that the raw
signal of two NO2-B42F sensors were dependent of the rate of change of
humidity with humidity changing between 40 and 60 \% at rate of about
Questions: should we use dynamic measurement or static for calibration,
what is the rate of change of humidity in field? Add about CO2 from
Lewis and Hueglin

Does sensitivity change with Temperature and humidity: use design of
experiment

\section{Theory}\label{theory}

\subsection{Electrochemical sensors}\label{electrochemical-sensors}

Amperometric gas sensors (AGS) sensors are based on electrochemical
cells assuring the transfer of charges from an electrode to another
which are in connection through an electrolyte phase. This electrolyte
phase has to carry the cell current by enabling the transport of charge
carriers in form of ions and often has to provide co-reactants (usually
water, protons or hydroxide ions) to the electrode as well as to allow
the removal of ionic products from the reaction site. The electrolyte
can be solid \citep{kumar_solid_2000}, gel-like or organic gel as in the
case of Sensoric City Technology sensors (DE)
\citep{xiong_amperometric_2014}, liquid or gaseous electrolytes e. g.
for sensors manufactured by SGX Sensortech (CH). AGSs require at least a
measuring electrode and a counter electrode through which flows a
current generated by the redox reaction of an analyte at the working
electrode when a fixed or variable proper potential is applied to sensor
electrodes. For measurement at low levels, the majority of amperometric
sensors include a \(3^{rd}\) electrode, the reference electrode, to
which a bias voltage can be applied to modify the potanital of the
working electrode. More recently a \(4^th\) electrode, named an
auxiliary electrode, have been added to a few sensor model (e. g. the
sensors manufactured by AlphaSense (UK), CityTechnology Ltd (UK) and
Membrapor (CH)). The \(4^th\) electrode is used to correct for changes
in sensor signal that are not induced by changes in the gaseous compound
being sampled \citep{popoola_development_2016} + (American reference)
but it accounts for changes in for example temperature and humidity of
the ambient air being sampled.\\
In the AGSs, the reaction rate, reflected by the current, \(i_{lim}\),
at the sensing electrode, occurs for any reaction triggered by the
applied electrode potential. The fundamental process for sensing an
analyte by an AGS can be described in four steps: (1) the analyte
diffuses to the sensing electrode. In order to achieve selectivity
and/or diffusion-limited working mode this diffusion may proceed through
a membrane or some other diffusion barrier. (2) The analyte is adsorbed
on the sensing electrode. (3) The electrochemical reaction occurs. (4)
The reaction products desorb from the sensing electrode and diffuse away
\citep{helm_measurement_2010}. Using Fick's first law and Faraday's law
the following general expression for the steady state current of an
amperometric sensor can be written \citep{helm_measurement_2010}:

\begin{equation}
i_{lim} = \frac {nF}{ R_k + \sum_{i=1}^{n} R_i} [C]_{gas} 
\label{ilim}
\end{equation}

where \(i_{lim}\) is the sensor output current, n is number of electrons
involved in electrochemical reaction, F is the Faraday constant, \(R_k\)
is the kinetic resistance of the electrochemical reaction and \(R_i\)
are several resistances to the analyte diffusion at layers i. ADD the
layer from Popoola, 2016 Generally a membrane in placed on top of the
working electrode so that the rate of mass transport (\(R_d\)) by
diffusion and permeation through the membrane of the reactant to the
electrode surface is much higher than the rate of the electrode reaction
(\(R_k\)). In fact, if \(R_k\) is much higher than \(\sum R_i\), the
rate of reaction at the electrode surface is the rate-limiting step, the
limiting current, \(i_{lim}\) is controlled by the rate of the electrode
kinetics. In this case, the analyte reaches the surface much faster than
it is reacted, and so the concentration at the electrode surface is the
same as in the gas surrounding the electrode. When operated under
appropriate diffusion-limited conditions, \(i_{lim}\), is simply
proportional to the concentration of the analyte
\citep{helm_measurement_2010} as shown in Equation \eqref{ilim_linear}.
The simple relationship given in Equation \eqref{ilim_linear} is valid
provided that the limiting factor to the transport of charge in the
sensor cell is controlled by the gas molecules diffusing to the working
electrode rather than by the rate of redox reaction at this electrode.
In this equation, the current, ilim, is directly proportional to, C, the
gaseous concentration in volume or mass concentration. This is only
valid for steady concentration using the 1st Fick's law, what about 2nd
law of Fick when concentration is changing?

\begin{equation}
i_{lim} =  k [C]_{gas} 
\label{ilim_linear}
\end{equation}

The expression for the current obeys the Faraday's law and a dynamic
reaction achieving a steady-state condition in the sensor and it takes
the form of Equation \eqref{ilim_exp} where k represents the standard
rate constant, F is the faraday constant, R is the gas constant, T is
the Kelvin temperature, A is the area of the electrode, C is the
concentration, n represents the number of electrons per molecule
reacting, and R and E are the transfer coefficient and overvoltage of
the electrode reaction, respectively. Although Equation \eqref{ilim_exp}
corresponds to a limited current state that is not used in this study,
it shows what can be the influence of temperature in the sensor cell.

\begin{equation}
i_{lim} =  nFkA e^{\frac {nFE}{RT}} 
\label{ilim_exp}
\end{equation}

This process comprises a chemical reaction of the electrode as well as
charge transport through the electrolyte, which can both be chemically
influenced. amperometric if the current of an oxidoreduction reaction
that is linearly proportional to the gas concentration is measured. The
principle behind amperometric sensors is the measurement of the
current-potential relationship in an electrochemical cell where
equilibrium is not established. The current is quantitatively related to
the rate of the electrolytic process at the working electrode whose
potential is kept constant using the so-called reference electrode. The
gas molecules diffuses into the sensor and the measuring electrode where
a direct electron transfer takes place due to chemical reactions. These
reactions produce a current proportional to the concentration of the
compound \citep{knake_amperometric_2005} following the Nernst Law.
Nowadays, the amperometric sensors also includes a reference electrodes,
while the trend is to add a 4th auxiliary electrode for correction of
electrodes physical changes and sensor drift. Hereafter we present an
evaluation of commercially available sensors for O3 and NO2.

\subsection{Calibration of sensors}\label{calibration-of-sensors}

The objective is to set calibration equations that fit sensor raw
current over a short calibration time interval when sensor and reference
data are available. This is called calibration as in Equation
\eqref{Ri_cal}, where \(R_i\) is the raw CO, NO, NO2 or O3 sensor data
in nA or Volt that is fitted as a multilinear combination of \(C_j\),
the set of parameters that affect the sensor measurements at degree k
(integer or real) and \(a_j\) the coefficients of the multilinear
combination. One should note that for j equals 1, \(C_1\) is the
Identity vector and \(a_1\) is the baseline or zero current, and for j
equals 2, \(C_2\) is \(C_i\) the reference gas concentrations of
interest while \(a_2\) is the sensitivity of the sensor for the gas
compounds being monitored. For j \textgreater{} 2, \(C_j\) can be called
covariates.

\begin{equation}
R_i = \sum_{i=1}^{n} a_j [C_j]^k = a_1 + a_2[C_i] + \sum_{i=3}^{n} a_j [C_j]^k 
\label{Ri_cal}
\end{equation}

Subsequently the calibration is applied to compute pollutant levels
outside the calibration time period using Equation \eqref{Ci_cal},
hereafter called prediction or inference.

\begin{equation}
[C_i] = \frac {R_i-a_1 -(\sum_{i=3}^{n} a_j [C_j]^k)}{a_2}
\label{Ci_cal}
\end{equation}

\section{Materials and Methods}\label{materials-and-methods}

Materials and Methods should be described with sufficient details to
allow others to replicate and build on published results. Please note
that publication of your manuscript implicates that you must make all
materials, data, computer code, and protocols associated with the
publication available to readers. Please disclose at the submission
stage any restrictions on the availability of materials or information.
New methods and protocols should be described in detail while
well-established methods can be briefly described and appropriately
cited.

Research manuscripts reporting large datasets that are deposited in a
publicly available database should specify where the data have been
deposited and provide the relevant accession numbers. If the accession
numbers have not yet been obtained at the time of submission, please
state that they will be provided during review. They must be provided
prior to publication.

Interventionary studies involving animals or humans, and other studies
require ethical approval must list the authority that provided approval
and the corresponding ethical approval code.

\section{Results}\label{results}

This section may be divided by subheadings. It should provide a concise
and precise description of the experimental results, their
interpretation as well as the experimental conclusions that can be
drawn.

\subsection{Subsection Heading Here}\label{subsection-heading-here}

Subsection text here.

\subsubsection{Subsubsection Heading
Here}\label{subsubsection-heading-here}

Bulleted lists look like this:

\begin{itemize}
\tightlist
\item
  First bullet
\item
  Second bullet
\item
  Third bullet
\end{itemize}

Numbered lists can be added as follows:

\begin{enumerate}
\def\labelenumi{\arabic{enumi}.}
\tightlist
\item
  First item
\item
  Second item
\item
  Third item
\end{enumerate}

The text continues here.

All figures and tables should be cited in the main text as Figure 1,
Table 1, etc.

\begin{figure}[H]
\centering
\includegraphics[width=3 cm]{logo-mdpi}
\caption{This is a figure, Schemes follow the same formatting. If there are multiple panels, they should be listed as: (\textbf{a}) Description of what is contained in the first panel. (\textbf{b}) Description of what is contained in the second panel. Figures should be placed in the main text near to the first time they are cited. A caption on a single line should be centered.}
\end{figure}

\begin{table}[H]
\caption{This is a table caption. Tables should be placed in the main text near to the first time they are cited.}
\centering
%% \tablesize{} %% You can specify the fontsize here, e.g.  \tablesize{\footnotesize}. If commented out \small will be used.
\begin{tabular}{ccc}
\toprule
\textbf{Title 1}    & \textbf{Title 2}  & \textbf{Title 3}\\
\midrule
entry 1     & data          & data\\
entry 2     & data          & data\\
\bottomrule
\end{tabular}
\end{table}

This is an example of an equation:

\begin{equation}
\mathbb{S}
\end{equation}

Example of a theorem:

\begin{Theorem}
Example text of a theorem.
\end{Theorem}

The text continues here. Proofs must be formatted as follows:

Example of a proof:

\begin{proof}[Proof of Theorem 1]
Text of the proof. Note that the phrase `of Theorem 1' is optional if it is clear which theorem is being referred to.
\end{proof}

The text continues here.

\section{Discussion}\label{discussion}

Authors should discuss the results and how they can be interpreted in
perspective of previous studies and of the working hypotheses. The
findings and their implications should be discussed in the broadest
context possible. Future research directions may also be highlighted.

\section{Conclusion}\label{conclusion}

This section is not mandatory, but can be added to the manuscript if the
discussion is unusually long or complex.

% %%%%%%%%%%%%%%%%%%%%%%%%%%%%%%%%%%%%%%%%%%
% %% optional
% \supplementary{The following are available online at www.mdpi.com/link, Figure S1: title, Table S1: title, Video S1: title.}
%
% % Only for the journal Methods and Protocols:
% % If you wish to submit a video article, please do so with any other supplementary material.
% % \supplementary{The following are available at www.mdpi.com/link: Figure S1: title, Table S1: title, Video S1: title. A supporting video article is available at doi: link.}

\vspace{6pt}

%%%%%%%%%%%%%%%%%%%%%%%%%%%%%%%%%%%%%%%%%%
\acknowledgments{All sources of funding of the study should be disclosed. Please clearly
indicate grants that you have received in support of your research work.
Clearly state if you received funds for covering the costs to publish in
open access.}

%%%%%%%%%%%%%%%%%%%%%%%%%%%%%%%%%%%%%%%%%%
\authorcontributions{For research articles with several authors, a short paragraph specifying
their individual contributions must be provided. The following
statements should be used ``X.X. and Y.Y. conceive and designed the
experiments; X.X. performed the experiments; X.X. and Y.Y. analyzed the
data; W.W. contributed reagents/materials/analysis tools; Y.Y. wrote the
paper.'' Authorship must be limited to those who have contributed
substantially to the work reported.}

%%%%%%%%%%%%%%%%%%%%%%%%%%%%%%%%%%%%%%%%%%
\conflictsofinterest{The authors declare no conflict of interest.}

%%%%%%%%%%%%%%%%%%%%%%%%%%%%%%%%%%%%%%%%%%
%% optional
\abbreviations{The following abbreviations are used in this manuscript:\\

\noindent
\begin{tabular}{@{}ll}
MDPI & Multidisciplinary Digital Publishing Institute \\
DOAJ & Directory of open access journals \\
TLA & Three letter acronym \\
LD & linear dichroism \\
\end{tabular}}


%%%%%%%%%%%%%%%%%%%%%%%%%%%%%%%%%%%%%%%%%%
% Citations and References in Supplementary files are permitted provided that they also appear in the reference list here.

%=====================================
% References, variant A: internal bibliography
%=====================================
%\reftitle{References}
%\begin{thebibliography}{999}
% Reference 1
%\bibitem[Author1(year)]{ref-journal}
%Author1, T. The title of the cited article. {\em Journal Abbreviation} {\bf 2008}, {\em 10}, 142--149.
% Reference 2
%\bibitem[Author2(year)]{ref-book}
%Author2, L. The title of the cited contribution. In {\em The Book Title}; Editor1, F., Editor2, A., Eds.; Publishing House: City, Country, 2007; pp. 32--58.
%\end{thebibliography}

% The following MDPI journals use author-date citation: Arts, Econometrics, Economies, Genealogy, Humanities, IJFS, JRFM, Laws, Religions, Risks, Social Sciences. For those journals, please follow the formatting guidelines on http://www.mdpi.com/authors/references
% To cite two works by the same author: \citeauthor{ref-journal-1a} (\citeyear{ref-journal-1a}, \citeyear{ref-journal-1b}). This produces: Whittaker (1967, 1975)
% To cite two works by the same author with specific pages: \citeauthor{ref-journal-3a} (\citeyear{ref-journal-3a}, p. 328; \citeyear{ref-journal-3b}, p.475). This produces: Wong (1999, p. 328; 2000, p. 475)

%=====================================
% References, variant B: external bibliography
%=====================================
\reftitle{References}
\externalbibliography{yes}
\bibliography{mybibfile.bib}

%%%%%%%%%%%%%%%%%%%%%%%%%%%%%%%%%%%%%%%%%%
%% optional
\sampleavailability{Samples of the compounds \ldots{}\ldots{} are available from the
authors.}

%% for journal Sci
%\reviewreports{\\
%Reviewer 1 comments and authors’ response\\
%Reviewer 2 comments and authors’ response\\
%Reviewer 3 comments and authors’ response
%}

%%%%%%%%%%%%%%%%%%%%%%%%%%%%%%%%%%%%%%%%%%
\end{document}

